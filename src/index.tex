\maketitle

\begin{info}{Catatan Penulis}
  Semua rancangan aplikasi dalam bab-bab selanjutnya merupakan
  aplikasi pribadi \textit{open source} dengan sumber kode terbuka
  yang dapat diakses melalui alamat spesifik masing-masing aplikasi
\end{info}

\section{Rancangan Aplikasi Pelacakan Port}

Aplikasi ini merupakan situs web \url{https://phantomports.com} yang
berfungsi untuk melacak port\footnote{
  Port di jaringan komputer adalah identifikasi unik layanan atau
  aplikasi dalam sebuah perangkat. (e.g. server web HTTP biasanya
    berjalan di port \texttt{80}, sedangkan SSH biasanya berjalan di
  port \texttt{22})
}
yang belum terdaftar pada daftar layanan IANA\footnote{
  IANA (Internet Assigned Numbers Authority) adalah salah satu
  lembaga standardisasi internet yang  menyimpan daftar port yang
  telah digunakan oleh berbagai layanan di
  \url{https://www.iana.org/assignments/service-names-port-numbers/service-names-port-numbers.xhtml?search=90}
}. Tujuan utama dari aplikasi ini adalah untuk menyederhanakan proses
pencarian port untuk pengembang aplikasi lain agar tidak terjadi
pembentrokan port yang digunakan dengan aplikasi yang telah terdaftar
di daftar IANA.

Analisa model kognitif untuk aplikasi ini adalah proses pelacakan
port yang dilakukan oleh pengguna aplikasi.

\begin{info}{Catatan Penulis}
  Sumber kode aplikasi ini berada di repository berikut:
  \url{https://github.com/arsmoriendy/phantomports-front}
\end{info}

\clearpage
\subsection{Analisa Model Kognitif GOMS}

\begin{verbatim}
GOAL: LACAK-PORT
. [select GOAL: METODE-INPUT-BOX
.         . FOKUS-INPUT-BOX
.         . KETIK-PORT
.         . [select GOAL: METODE-KUNCI
.         .         . TEKAN-ENTER
.         .         GOAL: METODE-TOMBOL
.         .         . TEKAN-TOMBOL]
.         GOAL: METODE-PARAMETER-QUERY
.         . FOKUS-ALAMAT
.         . KETIK-PREFIX
.         . KETIK-PORT
.         . TEKAN-ENTER]
. VERIFIKASI-PORT
\end{verbatim}

\begin{gomsfig}
  \begin{gomstbl}
    \texttt{LACAK-PORT} & Goal & Melacak jika port tertentu telah
    terdaftar pada daftar IANA \\
    \texttt{METODE-INPUT-BOX} & Method & Tentukan port yang ingin
    dilacak menggunakan kotak input port \\
    \texttt{FOKUS-INPUT-BOX} & Operator & Pastikan fokus berada pada
    kotak input port \\
    \texttt{KETIK-PORT} & Operator & Ketik port yang ingin dilacak \\
    \texttt{METODE-KUNCI} & Method & Tekan kunci keyboard tertentu
    untuk mengeksekusi pencarian port yang telah ditentukan \\
    \texttt{TEKAN-ENTER} & Operator & Tekan kunci \Enter pada keyboard \\
    \texttt{METODE-TOMBOL} & Method & Tekan tombol untuk mengeksekusi
    pencarian port yang telah ditentukan \\
    \texttt{TEKAN-TOMBOL} & Operator & Tekan tombol ``Search'' \\
    \texttt{METODE-PARAMETER-QUERY} & Method & Tentukan port yang ingin
    dilacak melalui parameter query\footnotemark di \textit{address
    bar}\footnotemark \\
    \texttt{FOKUS-ALAMAT} & Operator & Pastikan fokus berada pada
    \textit{address bar} \\
    \texttt{KETIK-PREFIX} & Operator & Ketik bagian alamat awal (i.e.
      sebelum nilai kunci parameter query port:
    \texttt{https://phantomports.com/?port=}) \\
    \texttt{VERIFIKASI-PORT} & Operator & Pastikan port yang
    ditampilkan sesuai dengan yang dilacak \\
  \end{gomstbl}
  \caption{Penjelasan GOMS aplikasi pencarian port}
\end{gomsfig}

\footnotetext[3]{
  Terjemahan dari \textit{query parameter}, yaitu salah satu bagian
  alamat situs web yang dapat menentukan identifikasi sumber daya
  spesifik. Format dimulai dengan karakter \texttt{?} dan berisi
  pasangan kunci dan nilai (e.g. untuk menyetel kunci parameter
    \texttt{port} ke 22, pengguna dapat menuliskan alamat:
  \texttt{https://phantomports.com/\fbox{?port=22}})
}
\footnotetext[4]{Kotak alamat situs web, yang biasanya terletak di
bagian atas sebuah \textit{web browser}}

\subsection{Analisa Model Kognitif KLM}

\begin{table}[H]
  \centering
  \begin{tabularx}{\columnwidth}{lXlX}
    \hline
    \multicolumn{2}{c}{\textbf{\texttt{METODE-KUNCI}}} &
    \multicolumn{2}{c}{\textbf{\texttt{METODE-TOMBOL}}} \\
    \hline

    H[keyboard] & $0.4$ & H[mouse] & $0.4$ \\
    K[\Enter] & $0.1$ & P[ke tombol ``Search''] & $0.5$ \\
    && B[tekan kiri] & $0.1$ \\

    \hline
    \textbf{Total} & $0.5$ detik & \textbf{Total} & $1.0$ detik \\
    \hline
  \end{tabularx}
  \caption{Perbandingan model kognitif KLM antara
    \texttt{METODE-KUNCI} dengan asumsi posisi tangan berawal di mouse,
  dan \texttt{METODE-TOMBOL} dengan asumsi posisi tangan berawal di keyboard}
\end{table}

\begin{table}[H]
  \centering
  \begin{tabularx}{\columnwidth}{lXlX}
    \hline
    \multicolumn{2}{c}{\textbf{\texttt{METODE-INPUT-BOX}}} &
    \multicolumn{2}{c}{\textbf{\texttt{METODE-PARAMETER-QUERY}}} \\
    \hline

    H[mouse] & $0.4$ & H[mouse] & $0.4$ \\
    P[kotak input] & $0.5$ & P[\textit{address bar}] & $0.5$ \\
    B[tekan kiri] & $0.1$ & B[tekan kiri] & $0.1$ \\
    K[port] & $0.2$ - $1.0$ & M[prefix] & $0.7$ \\
    \makecell{\texttt{METODE-KUNCI}/\\\texttt{METODE-TOMBOL}} & $0.5$ - $1.0$ &
    K[prefix] & $3.0$ - $9.0$ \\
    & & K[port] & $0.2$ - $1.0$ \\
    & & K[\Enter] & $0.1$ \\

    \hline
    \textbf{Total} & $1.7$ - $3.0$ detik & \textbf{Total} & $5.0$ -
    $11.8$ detik \\
    \hline
  \end{tabularx}
  \caption{Perbandingan model kognitif KLM antara
    \texttt{METODE-INPUT-BOX} dan \texttt{METODE-PARAMETER-QUERY} dengan asumsi
    posisi tangan berawal di keyboard dan pengguna telah berada di
  halaman aplikasi}
\end{table}

\begin{table}[H]
  \centering
  \begin{tabularx}{\columnwidth}{lX}
    \hline
    \multicolumn{2}{c}{\textbf{\texttt{LACAK-PORT}}} \\
    \hline

    \makecell{\texttt{METODE-INPUT-BOX}/\\\texttt{METODE-PARAMETER-QUERY}}
    & $1.7$ - $11.8$ \\
    R & $0.1$ - $5$ \\
    M[verifikasi] & $0.4$ \\

    \hline
    \textbf{Total} & $2.2$ - $17.2$ detik \\
    \hline
  \end{tabularx}
  \caption{Model kognitif KLM untuk \texttt{LACAK-PORT}}
\end{table}
