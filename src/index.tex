\maketitle

\begin{info}{Catatan Penulis}
  Semua rancangan aplikasi dalam bab-bab selanjutnya merupakan
  aplikasi pribadi \textit{open source} dengan sumber kode terbuka
  yang dapat diakses melalui alamat spesifik masing-masing aplikasi
\end{info}

\section{Rancangan Aplikasi Pelacakan Port}

Aplikasi ini merupakan situs web \url{https://phantomports.com} yang
berfungsi untuk melacak port\footnote{
  Port di jaringan komputer adalah identifikasi unik layanan atau
  aplikasi dalam sebuah perangkat. (e.g. server web HTTP biasanya
    berjalan di port \texttt{80}, sedangkan SSH biasanya berjalan di
  port \texttt{22})
}
yang belum terdaftar pada daftar layanan IANA\footnote{
  IANA (Internet Assigned Numbers Authority) adalah salah satu
  lembaga standardisasi internet yang  menyimpan daftar port yang
  telah digunakan oleh berbagai layanan di
  \url{https://www.iana.org/assignments/service-names-port-numbers/service-names-port-numbers.xhtml?search=90}
}. Tujuan utama dari aplikasi ini adalah untuk menyederhanakan proses
pencarian port untuk pengembang aplikasi lain agar tidak terjadi
pembentrokan port yang digunakan dengan aplikasi yang telah terdaftar
di daftar IANA.

Analisa model kognitif untuk aplikasi ini adalah proses pelacakan
port yang dilakukan oleh pengguna aplikasi.

\begin{info}{Catatan Penulis}
  Sumber kode aplikasi ini berada di repository berikut:
  \url{https://github.com/arsmoriendy/phantomports-front}
\end{info}

\clearpage
\subsection{Analisa Model Kognitif GOMS}

\begin{verbatim}
GOAL: LACAK-PORT
. [select GOAL: METODE-INPUT-BOX
.         . FOKUS-INPUT-BOX
.         . KETIK-PORT
.         . [select GOAL: METODE-KUNCI
.         .         . TEKAN-ENTER
.         .         GOAL: METODE-TOMBOL
.         .         . TEKAN-TOMBOL]
.         GOAL: METODE-PARAMETER-QUERY
.         . FOKUS-ALAMAT
.         . KETIK-PREFIX
.         . KETIK-PORT
.         . TEKAN-ENTER]
. VERIFIKASI-PORT
\end{verbatim}

\begin{chtbl}
  \begin{gomstbl}
    \gomsrow{LACAK-PORT}G{
      Melacak jika port tertentu telah terdaftar pada daftar IANA
    }

    \gomsrow{METODE-INPUT-BOX}M{
      Tentukan port yang ingin dilacak menggunakan kotak input port
    }

    \gomsrow{FOKUS-INPUT-BOX}O{
      Pastikan fokus berada pada kotak input port
    }

    \gomsrow{KETIK-PORT}O{
      Ketik port yang ingin dilacak
    }

    \gomsrow{METODE-KUNCI}M{
      Tekan kunci keyboard tertentu untuk mengeksekusi pencarian port
      yang telah ditentukan
    }

    \gomsrow{TEKAN-ENTER}O{
      Tekan kunci \Enter pada keyboard
    }

    \gomsrow{METODE-TOMBOL}M{
      Tekan tombol untuk mengeksekusi pencarian port yang telah ditentukan
    }

    \gomsrow{TEKAN-TOMBOL}O{
      Tekan tombol ``Search''
    }

    \gomsrow{METODE-PARAMETER-QUERY}M{
      Tentukan port yang ingin dilacak melalui parameter
      query\footnotemark di \textit{address bar}\footnotemark
    }

    \gomsrow{FOKUS-ALAMAT}O{
      Pastikan fokus berada pada \textit{address bar}
    }

    \gomsrow{KETIK-PREFIX}O{
      Ketik bagian alamat awal (i.e. sebelum nilai kunci parameter
      query port: \texttt{https://phantomports.com/?port=})
    }

    \gomsrow{VERIFIKASI-PORT}O{
      Pastikan port yang ditampilkan sesuai dengan yang dilacak
    }
  \end{gomstbl}
  \caption{Penjelasan GOMS aplikasi pencarian port}
\end{chtbl}

\footnotetext[3]{
  Terjemahan dari \textit{query parameter}, yaitu salah satu bagian
  alamat situs web yang dapat menentukan identifikasi sumber daya
  spesifik. Format dimulai dengan karakter \texttt{?} dan berisi
  pasangan kunci dan nilai (e.g. untuk menyetel kunci parameter
    \texttt{port} ke 22, pengguna dapat menuliskan alamat:
  \texttt{https://phantomports.com/\fbox{?port=22}})
}
\footnotetext[4]{Kotak alamat situs web, yang biasanya terletak di
bagian atas sebuah \textit{web browser}}

\subsection{Analisa Model Kognitif KLM}

\begin{chtbl}
  \begin{klmtbl}{lXlX}
    \klmheader{
      \multicolumn{2}{c}{\klmhead{METODE-KUNCI}} &
      \multicolumn{2}{c}{\klmhead{METODE-TOMBOL}}
    }

    H[keyboard] & $0.4$ & H[mouse] & $0.4$ \\
    K[\Enter] & $0.1$ & P[ke tombol ``Search''] & $0.5$ \\
    && B[tekan kiri] & $0.1$ \\

    \klmfooter{
      \klmtotal{$0.5$} & \klmtotal{$1.0$}
    }
  \end{klmtbl}
  \caption{Perbandingan model kognitif KLM antara
    \texttt{METODE-KUNCI} dengan asumsi posisi tangan berawal di mouse,
  dan \texttt{METODE-TOMBOL} dengan asumsi posisi tangan berawal di keyboard}
\end{chtbl}

\begin{chtbl}
  \begin{klmtbl}{lXlX}
    \klmheader{
      \multicolumn{2}{c}{\klmhead{METODE-INPUT-BOX}} &
      \multicolumn{2}{c}{\klmhead{METODE-PARAMETER-QUERY}}
    }

    H[mouse] & $0.4$ & H[mouse] & $0.4$ \\
    P[kotak input] & $0.5$ & P[\textit{address bar}] & $0.5$ \\
    B[tekan kiri] & $0.1$ & B[tekan kiri] & $0.1$ \\
    K[port] & $0.2$ - $1.0$ & M[prefix] & $0.7$ \\
    \makecell{\texttt{METODE-KUNCI}/\\\texttt{METODE-TOMBOL}} & $0.5$ - $1.0$ &
    K[prefix] & $3.0$ - $9.0$ \\
    & & K[port] & $0.2$ - $1.0$ \\
    & & K[\Enter] & $0.1$ \\

    \klmfooter{
      \klmtotal{$1.7$ - $3.0$} & \klmtotal{$5.0$ - $11.8$}
    }
  \end{klmtbl}
  \caption{Perbandingan model kognitif KLM antara
    \texttt{METODE-INPUT-BOX} dan \texttt{METODE-PARAMETER-QUERY} dengan asumsi
    posisi tangan berawal di keyboard dan pengguna telah berada di
  halaman aplikasi}
\end{chtbl}

\begin{chtbl}
  \begin{klmtbl}{XX}
    \klmheader{
      \multicolumn{2}{c}{\klmhead{LACAK-PORT}}
    }

    \makecell{\texttt{METODE-INPUT-BOX}/\\\texttt{METODE-PARAMETER-QUERY}}
    & $1.7$ - $11.8$ \\
    R & $0.1$ - $5$ \\
    M[verifikasi] & $0.4$ \\

    \klmfooter{
      \klmtotal{$2.2$ - $17.2$}
    }
  \end{klmtbl}
  \caption{Model kognitif KLM untuk \texttt{LACAK-PORT}}
\end{chtbl}

\section{Rancangan Aplikasi \textit{Cheatsheet} Tema Warna}

Analisa selanjutnya akan dilakukan pada aplikasi
\url{https://arsmoriendy.github.io/gruvbox-cheatsheet}. Situs web
ini bertujuan untuk menyajikan beragam macam format kode warna (e.g.
format RGB, HSL, HEX) dari tema warna bernama Gruvbox\footnote{
  Tema warna Gruvbox berawal sebagi ekstensi tema warna untuk
  program editor teks Vim \url{https://github.com/morhetz/gruvbox}
}.

\marginnote{
  \faPalette\ Dokumen ini, serta semua aplikasi yang didiskusikan,
  merupakan contoh implementasi tema warna Gruvbox
}

Pada aplikasi ini, terdapat fitur penyalinan kode-kode warna dalam
bentuk JSON. Proses inilah yang akan dianalisa pada subbab-subbab selanjutnya.

\begin{info}{Catatan Penulis}
  Sumber kode aplikasi ini berada di repository GitHub berikut:
  \url{https://github.com/arsmoriendy/gruvbox-cheatsheet}
\end{info}

\clearpage
\subsection{Analisa Model Kognitif GOMS}

\begin{verbatim}
GOAL: SALIN-JSON
. TEKAN-TOMBOL-DIALOG
. [select GOAL: METODE-TOMBOL
.         . TEKAN-TOMBOL-SALIN
.         GOAL: METODE-HIGHLIGHT
.         . HIGHLIGHT-JSON
.         . [select GOAL: METODE-MENU
.         .         . KLIK-KANAN
.         .         . KLIK-SALIN
.         .         GOAL: METODE-CTRL
.         .         . TEKAN-CTRL-C]]
\end{verbatim}

\begin{chtbl}
  \begin{gomstbl}
    \gomsrow{SALIN-JSON}G{
      Salin kode-kode warna berbentuk JSON
    }

    \gomsrow{TEKAN-TOMBOL-DIALOG}O{
      Tekan tombol (dengan simbol ``\verb`{}`'') untuk menampilkan
      modal dialog kode-kode warna JSON
    }

    \gomsrow{METODE-TOMBOL}M{
      Tekan tombol untuk menyalin JSON
    }

    \gomsrow{TEKAN-TOMBOL-SALIN}O{
      Tekan tombol (dengan simbol ``\faCopy[regular]'') untuk menyalin JSON
    }

    \gomsrow{METODE-HIGHLIGHT}M{
      \textit{Highlight} untuk menyalin seluruh kode JSON
    }

    \gomsrow{HIGHLIGHT-JSON}O{
      \textit{Highlight} seluruh kode JSON
    }

    \gomsrow{METODE-MENU}M{
      Gunakan menu konteks untuk menyalin teks
    }

    \gomsrow{KLIK-KANAN}O{
      Tekan tombol kanan mouse
    }

    \gomsrow{KLIK-SALIN}O{
      Tekan pilihan salin atau ``Copy'''
    }

    \gomsrow{METODE-CTRL}M{
      Gunakan \textit{shortcut} keyboard yang berawal dengan \Ctrl
    }

    \gomsrow{TEKAN-CTRL-C}O{
      Tekan \Ctrl{+}\keystroke{C} pada keyboard
    }
  \end{gomstbl}
  \caption{Penjelasan GOMS aplikasi \textit{cheatsheet} tema warna}
\end{chtbl}

\clearpage
\subsection{Analisa Model Kognitif KLM}

\begin{chtbl}
  \begin{klmtbl}{lXlX}
    \klmheader{
      \multicolumn{2}{c}{\klmhead{METODE-CTRL}} &
      \multicolumn{2}{c}{\klmhead{METODE-MENU}}
    }

    H[keyboard] & $0.4$ & B[tekan kanan] & $0.1$ \\
    K[\Ctrl{+}\keystroke{c}] & $0.3$ & P[ke pilihan ``Copy''] & $0.7$ \\
    & & B[tekan kiri] & $0.1$ \\

    \klmfooter{
      \klmtotal{$0.7$} & \klmtotal{$0.9$}
    }
  \end{klmtbl}
  \caption{Perbandingan model kognitif KLM antara
    \texttt{METODE-CTRL} dengan \texttt{METODE-MENU} dengan asumsi
    posisi tangan pengguna berawal di mouse, setelah melewati metode
  \texttt{METODE-HIGHLIGHT}}
\end{chtbl}

\begin{chtbl}
  \begin{klmtbl}{lXlX}
    \klmheader{
      \multicolumn{2}{c}{\klmhead{METODE-TOMBOL}} &
      \multicolumn{2}{c}{\klmhead{METODE-HIGHLIGHT}}
    }

    P[tombol ``\faCopy[regular]''] & $0.3$ & P[awal kode JSON] & $0.3$ \\
    B[tekan kiri] & $0.1$ & B[tekan kiri] & $0.1$ \\
    & & D[akhir kode JSON] & $1.4$ \\
    & & \makecell{
      \texttt{METODE-CTRL}/\\\texttt{METODE-MENU}
    } & $0.7$ - $0.9$ \\

    \klmfooter{
      \klmtotal{$0.4$} & \klmtotal{$2.5$ - $2.7$}
    }
  \end{klmtbl}
  \caption{
    Perbandingan model kognitif KLM antara \texttt{METODE-TOMBOL}
    dengan \texttt{METODE-HIGHLIGHT} dengan asumsi posisi tangan
    pengguna berawal di mouse, setelah melakukan operasi
    \texttt{TEKAN-TOMBOL-DIALOG}
  }
\end{chtbl}

\begin{chtbl}
  \begin{klmtbl}{XX}
    \klmheader{
      \multicolumn{2}{c}{\klmhead{SALIN-JSON}}
    }

    P[tombol dialog ``\verb`{}`''] & $0.2$ \\
    B[tekan kiri] & $0.1$ \\
    \makecell{
      \texttt{METODE-TOMBOL}/\\
      \texttt{METODE-HIGHLIGHT}
    } & $0.4$ - $2.7$ \\

    \klmfooter{
      \klmtotal{$0.7$ - $3.0$}
    }
  \end{klmtbl}
  \caption{
    Model kognitif KLM \texttt{SALIN-JSON} dengan asumsi posisi
    tangan pengguna berawal di mouse
  }
\end{chtbl}
